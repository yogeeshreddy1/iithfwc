\def\mytitle{ Design of Half subtractor using NOR gates}
\def\mykeywords{}
\def\myauthor{YOGEESH REDDY}
\def\contact{edamakantiyogeesh@gmail.com}
\def\mymodule{ Future Wireless Communication(FWC22073)}
\documentclass[10pt, a4paper]{article}
\usepackage[a4paper,outer=1.5cm,inner=1.5cm,top=1.75cm,bottom=1.5cm]{geometry}
\twocolumn
\usepackage{circuitikz}
\usepackage{graphicx}
\graphicspath{{./images/}}
%colour our links, remove weird boxes
\usepackage[colorlinks,linkcolor={black},citecolor={blue!80!black},urlcolor={blue!80!black}]{hyperref}
%Stop indentation on new paragraphs
\usepackage[parfill]{parskip}
%% Arial-like font
\usepackage{lmodern}
\renewcommand*\familydefault{\sfdefault}
%Napier logo top right
\usepackage{watermark}
%Lorem Ipusm dolor please don't leave any in you final report ;)
\usepackage{karnaugh-map} 
\usepackage{tabularx}
\usepackage{lipsum}
\usepackage{xcolor}
\usepackage{listings}
%give us the Capital H that we all know and love
\usepackage{float}
%tone down the line spacing after section titles
\usepackage{titlesec}
%Cool maths printing
\usepackage{amsmath}
%PseudoCode
\usepackage{algorithm2e}

\titlespacing{\subsection}{0pt}{\parskip}{-3pt}
\titlespacing{\subsubsection}{0pt}{\parskip}{-\parskip}
\titlespacing{\paragraph}{0pt}{\parskip}{\parskip}
\newcommand{\figuremacro}[5]{
    \begin{figure}[#1]
        \centering
        \includegraphics[width=#5\columnwidth]{#2}
        \caption[#3]{\textbf{#3}#4}
        \label{fig:#2}
    \end{figure}
}


 \lstset{
frame=single, 
breaklines=true,
columns=fullflexible
}

\thiswatermark{\centering \put(0,-100.0){\includegraphics[scale=0.05]{iith}} }
\title{\mytitle}
\author{\myauthor\hspace{1em}\\\contact\\IITH\hspace{0.5em}-\hspace{0.5em}\mymodule}
\date{}
\hypersetup{pdfauthor=\myauthor,pdftitle=\mytitle,pdfkeywords=\mykeywords}
\sloppy
% #######################################
% ########### START FROM HERE ###########
% #######################################
\begin{document}
 \maketitle
\begin{abstract}
   %Replace the lipsum command with actual text 
 \paragraph{ Half Subtractor (HS)}: Half subtractor is a combination circuit with two inputs and two outputs which is difference and borrow. It produces the difference between the two binary bits at the input and also produces an output (Borrow)
 \end{abstract}
\section{Components}
\begin{table}[htbp]
 \begin{center}
    \begin{tabular}{|l|c|c|c|c|} \hline
  \textbf{Component}& \textbf{Value} & \textbf{Quantity} \\
 \hline
 bread board& - & 1 \\ \hline
led &  - & 2 \\ \hline
Arduino & - & 1 \\ \hline
Jumper Wires & M-M & 2  \\ \hline
\end{tabular}  
\end{center}
\caption{\label{table:dummytable}}
\end{table}
 \section{Karnaugh-map}
 \subsection{truth table} 
 \begin{table}[htbp]
 \begin{center}
    \begin{tabular}{|l|c|c|c|c|} \hline
  \textbf{A}& \textbf{B} & \textbf{difference}& \textbf{borrow}\\
 \hline
 0&0&0&0\\ \hline
0&1&1&1 \\ \hline
1&0&1&0\\ \hline
1&1&0&0\\ \hline
\end{tabular}  
\end{center}
\caption{\label{table:dummytable} }
\end{table}
\subsection{Difference}
\begin{karnaugh-map} [2][2][1][$A$][$B$]
\minterms{1,2}
\maxterms{0,3}
\ implicantsol{2}
  \ implicantsol{1}
  \end{karnaugh-map}
  
  \begin{center}
      Figure 1:k-map
  \end{center}

From the above karnaugh-map the expression is A'B+AB'
    
    This karnaugh-map is verified by using Truth Table2
    
    \subsection{Borrow}
    \begin{karnaugh-map}[2][2][1][$B$][$A$]
    \minterms{1}
    \maxterms{0,2,3}
    \ implicantsol{2}
    \ implicantsol{1}
    \end{karnaugh-map}
    \begin{center}
        Figure 2:k-map
    \end{center}

  From the above karnaugh-map the expression is 
            A'B
            
    This karnaugh-map is verified by using Truth Table2        
        

\section{Circuit Diagram}
\begin{circuitikz} \draw
(-1,1) node[nor port] (nor1) {}
(nor1.in 1) -- ++ (-1,0) node[circ]{} node[left]{$A$}
(nor1.in 2) -- ++ (-1,0) node[circ]{} node[left]{$B$}
%(-1,1) node[nor port] (nor1) {}
(nor1.out) node(x1) [anchor=south west]{$x1$}
(1,2) node[nor port] (nor2) {}
(1,0) node[nor port] (nor3) {}
%(3,1) node[nor port] (nor4) {}

(5,1) node[nor port] (nor5) {}

(1,2) node[nor port] (nor2) {}  
(nor2.out) node(x2) [anchor=south west]{$BORROW =A'.B$}
(1,0) node[nor port] (nor3) {}
(nor3.out) node(x3) [anchor=north west] {$x3$}
(3,1) node[nor port] (nor4) {}
(nor4.out) node(x4) [anchor=south west] {$x4$}
%(4,1) node[nor port] (nor5) {}
(nor5.out) node(x5) [anchor=south west] {$DIFF =A'B+AB'$}


(nor2.out) -- (nor4.in 1)
(nor3.out) -- (nor4.in 2)
(nor1.out) -| (nor2.in 2)
(nor1.out) -| (nor3.in 1)
(nor1.in 1) |-(nor2.in 1)
(nor1.in 2) |-(nor3.in 2)
(nor4.out) -| (nor5.in 1)
(nor4.out) -| (nor5.in 2)

   ;\end{circuitikz}; 
\begin{center}
    Figure2
\end{center}



\begin{table}[htbp]


   
    \section{Hardware}
\begin{center}
    \begin{tabular}{|l|c|c|c|c|c|c|} \hline 
 \textbf{Arduino} & \textbf{D1,D2} & \textbf{GND} \\ \hline
 \textbf{Led} & \textbf{+VE} & \textbf{-VE}\\ \hline
 \textbf{Led} & \textbf{+VE} & \textbf{-VE}\\ \hline
\end{tabular}   
\end{center}
\caption{\label{table:dummytable}}
\end{table}
   
   \section{Hardware Connection}
   Give the connections as per Table 3. For taking the inputs connect 5V of arduino to +ve line of bread board to consider it as logic 'HIGH'.Connect GND pin of arduino to -ve line of bread board to consider it as logic 'LOW'.
\\
\\
For example if the inputs A,B are connected 1,0 respectively the output should be  i.e., the LED connected to the  2pin should turn on.
\\
\\
In the another case if we connect the inputs A,B to 1,1 respectively the output should be 0 i.e., the LED connected to 3th pin should off.

  The circuit implementation of the above function is given in figure 1.
\section{Software}
  1.Connect the arduino to the USB port of computer
  \\
  \\2.Download the follwing code
  \\
  \begin{lstlisting}
   https://github.com/yogeeshreddy1/iithfwc/blob/main/assignment/assignment.cpp
  \end{lstlisting}
  
  3.Upload the code into the arduino board.
  \\4.The output '1' is represented as the state:'LED ON' and '0' is represented as the state 'LED OFF'



   
   
 \end{document}